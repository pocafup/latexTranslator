\documentclass[11pt]{article}
\usepackage[margin=1in]{geometry}
\usepackage[T1]{fontenc}
\usepackage{lmodern}
\usepackage{amsmath,amssymb,mathtools}
\usepackage{tikz}
\usepackage{hyperref}
\setlength{\parskip}{0.6em}
\setlength{\parindent}{0pt}
\title{Translated Document}
\date{}
\begin{document}
\maketitle
% --- Page 1 ---
\clearpage
\section*{Page 1}


% --- Page 2 ---
\clearpage
\section*{Page 2}

```latex
\section*{Week 1: What is PDE?}

\subsection*{Analysis, Algebra, Geometry, Calculus}

\begin{align*}
    \text{PDE} &= \text{Partial Differential Equation} \\
    \text{ODE} &= \text{Ordinary Differential Equations}
\end{align*}

\subsection*{What is ODE?}

An equation that contains unknown function(s) of one variable.

\begin{equation}
    x + 2x + 3 = 5x
\end{equation}

\begin{equation}
    f(x) + \sin f(x) = 5 \quad \text{(Not an ODE)}
\end{equation}
```
% --- Page 3 ---
\clearpage
\section*{Page 3}

```latex
\section*{Ordinary Differential Equations (ODE)}

An ordinary differential equation (ODE) is an equation involving a function and its derivatives. For example, consider the equation:

\begin{equation}
f'(x) = \sin f(x)
\end{equation}

This is an ODE.

\section*{Partial Differential Equations (PDE)}

A partial differential equation (PDE) is an equation that contains unknown functions and their partial derivatives. For example, a first-order PDE can be written as:

\begin{equation}
E(x, y, u(x, y), u_x(x, y), u_y(x, y)) = 0
\end{equation}

or equivalently,

\begin{equation}
F(x, y, u, u_x, u_y) = 0
\end{equation}
```
% --- Page 4 ---
\clearpage
\section*{Page 4}

```latex
\section*{Examples of Partial Differential Equations (PDEs)}

\begin{align}
    &1. \quad U_x + U_y = 0 \quad \text{(Transport Equation)} \\
    &2. \quad U_x + y U_y = 0 \quad \text{(Transport Equation)} \\
    &3. \quad U_x + U U_y = 0 \quad \text{(Shock Wave Equation)} \\
    &4. \quad U_{xx} + U_{yy} = 0 \quad \text{(Laplace Equation)} \\
    &5. \quad U_{xx} + 4 = 0 \quad \text{(Wave Equation / Interactor)} \\
    &6. \quad U_t + U U_x + U_{xx} = 0 \quad \text{(Dispersive Equation)} \\
    &7. \quad U_{tt} + U_{xx} = 0 \quad \text{(Vibration of a Bar)} \\
    &8. \quad U - F U = 0 \quad \text{(Quantum Mechanics)} \\
    &\text{Parabolic Equation:} \quad U_t = U_{xx}
\end{align}
```
% --- Page 5 ---
\clearpage
\section*{Page 5}

```latex
\section*{Linear and Nonlinear Equations}

In linear algebra, a transformation \( T: V \to W \) is called a linear transformation if:

\begin{align}
    T(u + v) &= T(u) + T(v), \\
    T(au) &= aT(u).
\end{align}

Or equivalently,

\begin{equation}
    T(au + bv) = aT(u) + bT(v).
\end{equation}

\subsection*{Definition}

An operator is called linear if:

\begin{align}
    L(u + v) &= L(u) + L(v), \\
    L(au) &= aL(u).
\end{align}

A differential operator is linear if:

\begin{align}
    L(u + v) &= f(u) + f(v), \\
    L(au) &= af(x).
\end{align}

\subsection*{Example}

Consider the following examples:

\begin{align}
    L(u) &= 0, \\
    L(u) &= u_{xx} + u_{yy} = 0.
\end{align}

The transport equation is given by:

\begin{equation}
    u_x + u_y = 0.
\end{equation}

A linear partial differential equation (PDE) is written as:

\begin{equation}
    Lu = f(x).
\end{equation}

The system of linear equations can be represented as:

\begin{equation}
    Ax = b.
\end{equation}
```
% --- Page 6 ---
\clearpage
\section*{Page 6}

```latex
\section*{Solutions of Linear Equations}

Consider the linear equations:

\begin{align}
    Lu &= f(x), \\
    Ax &= b.
\end{align}

For the homogeneous case, we have:

\begin{align}
    Qu &= 0, \\
    Ax &= 0.
\end{align}

The general solution can be expressed as:

\begin{align}
    U &= U_g + U_s, \\
    X &= X_g + X_s,
\end{align}

where:
\begin{itemize}
    \item $U_g$ is a general solution of $Lu = 0$,
    \item $U_s$ is a special solution,
    \item $X_g$ is a general solution of $Ax = 0$,
    \item $X_s$ is a special solution.
\end{itemize}

\subsection*{Proposition}

Let $U_1, U_2, \ldots, U_n$ be solutions of $Lu = 0$. Then the linear combination

\begin{equation}
    u(t) = c_1 U_1 + c_2 U_2 + \cdots + c_n U_n
\end{equation}

is also a solution of $Lu = 0$, where $c_1, c_2, \ldots, c_n$ are constants.
```
% --- Page 7 ---
\clearpage
\section*{Page 7}

```latex
\section*{Example 1}

\begin{align}
    x(t) &= f(t) \\
    x(t) &= \int f(t) \, dt + C \\
    u(x) &= 0, \quad x = G \\
    u''(x) &= 0
\end{align}

\begin{equation}
    2(f) = C_1 + C_2
\end{equation}

\begin{align}
    u_{xx} &= 0 \\
    u(x, y) &= 0 \\
    u_x &= 0 \\
    u_x &= F(y) \\
    u(x, y) &= X F(y) + G(x)
\end{align}
```
% --- Page 8 ---
\clearpage
\section*{Page 8}

```latex
\section*{Example 2}

\begin{align}
    U_x X + U &= 0, \\
    u'' + u &= 0.
\end{align}

The solution for \( u(x, y) \) is given by:
\[
u(x, y) = C_1(y) \cos x + C_2(y) \sin x
\]
where \( u(x, y) = 0 \).

\section*{Example 3}

\begin{align}
    U_{xy} &= 0, \\
    u_{xx} &= 0, \\
    u_x &= F(x).
\end{align}

The general solution is:
\[
u = \int F(x) \, dx + H(x) + G(y).
\]
```
% --- Page 9 ---
\clearpage
\section*{Page 9}

```latex
\section*{List of Calculus Facts}

\begin{enumerate}
    \item Partial derivatives are local.
    
    \item $U_{xy} = K_{yx}$
    
    \item Chain rules:
    \begin{align*}
        f(g(x, y)) &= f'(g(x, y)) \cdot g'(x, y) \\
        &= f'(g(x, y)) \cdot \frac{\partial g}{\partial x} \quad \text{(special case of chain rule)}
    \end{align*}
    Assume $f = f(y_1, y_2, \ldots, y_n)$ and $y_i = g_i(x_1, \ldots, x_k)$ in the most general form of the chain rule.
    
    \item Green's formula (later):
    \[
    \int \left( \frac{\partial f}{\partial x} + \frac{\partial f}{\partial y} \right) \, dx \, dy = \oint f \, dt
    \]
    
    \item For a function $I(t)$:
    \[
    I(t) = \int_{a(t)}^{b(t)} f(x, t) \, dx
    \]
    The derivative is given by:
    \[
    I'(t) = f(b(t), t) \cdot b'(t) - f(a(t), t) \cdot a'(t) + \int_{a(t)}^{b(t)} \frac{\partial f}{\partial t} \, dx
    \]
\end{enumerate}
```
% --- Page 10 ---
\clearpage
\section*{Page 10}

```latex
\section*{Jacobian}

Consider the function \( F : \mathbb{R}^n \to \mathbb{R}^m \), where

\[
F = (F_1, F_2, \ldots, F_m)
\]

\section*{Infinite Series}

\emph{(to be covered later)}

\section*{Directional Derivative (Geometry)}

For a function \( u(x, y) \), the directional derivative in the direction of a vector \(\mathbf{u} = (u_1, u_2)\) is given by:

\begin{equation}
    \frac{\partial u}{\partial x} u_1 + \frac{\partial u}{\partial y} u_2
\end{equation}

This represents the rate of change of the function \( u \) in the direction of the vector \(\mathbf{u}\).

\section*{Familiar with Math 3D}

\emph{(to be covered later)}
```
% --- Page 11 ---
\clearpage
\section*{Page 11}

```latex
\section*{Example 1: Transport Equations}

Consider the transport equation:

\begin{equation}
    a u_x + b u_y = 0
\end{equation}

where \(a\) and \(b\) are constants.

\subsection*{Method 1: Geometry / Gradient of \(u\)}

The gradient of \(u\) is given by:

\[
(a, b) \cdot (u_x, u_y) = 0
\]

The equation for the line is:

\begin{equation}
    bx - ay = c
\end{equation}

The solution is:

\[
u(x, y) = F(bx - ay)
\]

To check:

\[
u_x = F' \cdot b, \quad u_y = F' \cdot (-a)
\]

Substituting back into the transport equation:

\[
a u_x + b u_y = a \cdot F' \cdot b + b \cdot F' \cdot (-a) = 0
\]

\subsection*{Method 2: Change of Variable}

Let:

\[
x' = ax + by, \quad y' = bx - ay
\]

% If a sketch is needed, it can be added here using TikZ.
```
% --- Page 12 ---
\clearpage
\section*{Page 12}

```latex
\section*{Example 2}

\begin{align}
    u_x &= Hy - a + Hyb = a \ell x + b u_y, \\
    My &= u_y b + Hy(-a) = b u_y - a n y, \\
    0 &= a \ell x + b u_y = (a + b) u_x = Hx = 0, \\
    x &= F(y) = F(bx - ay).
\end{align}

\begin{align}
    u_x + y u_y &= 0, \\
    2 &= + y^2 u = 0.
\end{align}

\begin{align}
    (u_x, u_y) &= 0, \\
    s I - (1, \%) &= 0.
\end{align}

\begin{align}
    y &= +, \\
    y &= c e^Y.
\end{align}
```
% --- Page 13 ---
\clearpage
\section*{Page 13}

```latex
\section*{Equations Along the Curve}

Consider the following equations along the curve:

\begin{align}
    u(x, Ex + x) &= 0, \\
    u(x, x + c) &= U_{10}, \\
    F(x) &= F(y - Ex) = F(bx - ay).
\end{align}
```
% --- Page 14 ---
\clearpage
\section*{Page 14}

```latex
\section*{Review}

\subsection*{Example}

Consider the equation:

\begin{align}
    ax + by &= 0.
\end{align}

If we consider the equation:

\begin{align}
    y &= Hx + My, \\
    y &= ux + My = 0,
\end{align}

we have:

\begin{align}
    u(x, x + c) &= u(0, c), \\
    u(x, y) &= u(0, c) = u(0, y - ax) = F(y - Ex) = G(ay - bx) = H(bx - ay).
\end{align}
```
% --- Page 15 ---
\clearpage
\section*{Page 15}

```latex
\section*{Example 2}

Consider the differential equation:
\begin{equation}
    u_x + y u_y = 0.
\end{equation}

Let \( y' = - \frac{u_x}{u_y} = y \). Then, the solution can be expressed as:
\begin{equation}
    y = ce^{x},
\end{equation}
where \( c \) is a constant. Thus, we have:
\begin{equation}
    u(x, y) = u(0, ce^{x}) = F(ce^{x}).
\end{equation}

\section*{Example ?}

In the previous example, if in addition \( u(0, y) = 3 \), then what is the solution?

\subsection*{Solution}

Given \( u(0, y) = 3 \), we have:
\begin{equation}
    u(x, y) = F(ye^{x}) = u(0, y) = 3.
\end{equation}
Thus, the solution is:
\begin{equation}
    u(x, y) = 3.
\end{equation}
```
\end{document}
