```latex
\section*{Exercise 7 of Section 1.2}

Solve the differential equation:
\begin{equation}
    Y U_x + X y = 0
\end{equation}
with the initial condition:
\begin{equation}
    x(0, y) = g
\end{equation}

Determine in which region of the initial problem (IP) the solution is uniquely determined.

\subsection*{Solution}

Consider the differential equation:
\begin{equation}
    y' = y
\end{equation}
which implies:
\begin{equation}
    y = x + c
\end{equation}
where \( c \) is a constant. Rewriting, we have:
\begin{equation}
    y - x = c
\end{equation}

If \( c = 0 \), then:
\begin{equation}
    y = x
\end{equation}

If \( c \neq 0 \), then:
\begin{equation}
    y^2 - x = 1
\end{equation}

If \( c = 0 \), then:
\begin{equation}
    y = x
\end{equation}

If \( c \neq 0 \), then:
\begin{equation}
    y = \ln U(x)
\end{equation}
where \( U(x) = u(x, t) = e \).

\begin{tikzpicture}
    \draw[->] (-1,0) -- (3,0) node[right] {$x$};
    \draw[->] (0,-1) -- (0,3) node[above] {$y$};
    \draw[domain=-1:2, smooth, variable=\x, blue] plot ({\x}, {\x}) node[right] {$y=x$};
    \draw[domain=-1:2, smooth, variable=\x, red] plot ({\x}, {sqrt(\x+1)}) node[right] {$y^2-x=1$};
\end{tikzpicture}
```
