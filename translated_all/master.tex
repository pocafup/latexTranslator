\documentclass[11pt]{article}
\usepackage[margin=1in]{geometry}
\usepackage[T1]{fontenc}
\usepackage{lmodern}
\usepackage{amsmath,amssymb,mathtools}
\usepackage{tikz}
\usepackage{hyperref}
\setlength{\parskip}{0.6em}
\setlength{\parindent}{0pt}
\title{PDE Notes (Translated)}
\date{}
\begin{document}
\maketitle

% --- Page 2 ---



\section*{Week 1}

\subsection*{What is PDE?}

\begin{itemize}
    \item PDE = Partial Differential Equation
    \item ODE = Ordinary Differential Equations
\end{itemize}

\subsection*{What is ODE?}

An equation that contains unknown function(s) of one variable.

\begin{align*}
    x + 2x + 3 &= 5 \\
    f(x) + \sin f(x) &= 5 \quad \text{(Not ODE)}
\end{align*}
% --- Page 3 ---



\section*{Ordinary Differential Equations (ODEs)}

An ordinary differential equation (ODE) is an equation involving a function and its derivatives. For example, consider the equation:

\begin{equation}
f'(x) = \sin f(x)
\end{equation}

This is an ODE.

\section*{Partial Differential Equations (PDEs)}

A partial differential equation (PDE) is an equation that contains unknown functions and their partial derivatives. For example, a first-order PDE can be written as:

\begin{equation}
E(x, y, u(x, y), u_x(x, y), u_y(x, y)) = 0
\end{equation}

where \( u(x, y) \) is a function of two variables \( x \) and \( y \), and \( u_x \) and \( u_y \) are the partial derivatives of \( u \) with respect to \( x \) and \( y \), respectively.

Another way to express a PDE is:

\begin{equation}
F(x, y, u, u_x, u_y) = 0
\end{equation}
% --- Page 4 ---




\section*{Examples of Partial Differential Equations (PDEs)}

\begin{align}
    1. \quad & U_x + U_y = 0 \quad \text{(Transport Equation)} \\
    2. \quad & U_x + y U_y = 0 \quad \text{(Transport Equation)} \\
    3. \quad & U_x + U U_y = 0 \quad \text{(Shock Wave Equation)} \\
    4. \quad & U_{xx} + U_{yy} = 0 \quad \text{(Laplace Equation)} \\
    5. \quad & U_{xx} + 4 = 0 \quad \text{(Wave Equation / Interactor)} \\
    6. \quad & U_t + U U_x + U_{xx} = 0 \quad \text{(Dispersive Equation)} \\
    7. \quad & U_{tt} + U_{xx} = 0 \quad \text{(Vibration of a Bar)} \\
    8. \quad & U - \nabla^2 U = 0 \quad \text{(Quantum Mechanics)} \\
    \text{Parabolic Equation:} \quad & U_t = U_{xx}
\end{align}

% --- Page 5 ---



\clearpage
\section*{Linear and Nonlinear Equations}

In linear algebra, a transformation \( T: V \to W \) is called a linear transformation if:

\begin{align}
    T(u + v) &= T(u) + T(v), \\
    T(au) &= aT(u)
\end{align}

or equivalently,

\begin{equation}
    T(au + bv) = aT(u) + bT(v)
\end{equation}

\subsection*{Definition}

An operator is called linear if:

\begin{align}
    L(u + v) &= L(u) + L(v), \\
    L(au) &= aL(u)
\end{align}

A differential operator is linear if:

\begin{align}
    L(u + v) &= L(u) + L(v), \\
    L(au) &= aL(u)
\end{align}

\subsection*{Example}

Consider the differential equation:

\begin{equation}
    u_{xx} + u_{yy} = 0
\end{equation}

This is an example of a linear partial differential equation (PDE).

The transport equation is given by:

\begin{equation}
    u_x + u_y = 0
\end{equation}

A linear PDE can be written in the form:

\begin{equation}
    Lu = f(x)
\end{equation}

where \( L \) is a linear operator, and \( f(x) \) is a given function.

The system of linear equations can be represented as:

\begin{equation}
    Ax = b
\end{equation}

% --- Page 6 ---




\section*{Solutions to Linear Equations}

Consider the linear equations:

\begin{align}
    Lu &= f(x), \\
    Ax &= b.
\end{align}

For the homogeneous case, we have:

\begin{align}
    Qu &= 0, \\
    Ax &= 0.
\end{align}

The general solution can be expressed as:

\begin{align}
    U &= U_g + U_s, \\
    X &= X_g + X_s,
\end{align}

where:
\begin{itemize}
    \item $U_g$ is a general solution of $Lu = 0$,
    \item $U_s$ is a special solution,
    \item $X_g$ is a general solution of $Ax = 0$,
    \item $X_s$ is a special solution of $Ax = b$.
\end{itemize}

\subsection*{Proposition}

Let $U_1, U_2, \ldots, U_n$ be solutions of $Lu = 0$. Then the linear combination

\begin{equation}
    u(t) = c_1 U_1 + c_2 U_2 + \cdots + c_n U_n
\end{equation}

is also a solution of $Lu = 0$, where $c_1, c_2, \ldots, c_n$ are constants.

% --- Page 7 ---




\section*{Example 1}

\begin{align}
    x(t) &= f(t) \\
    x(t) &= \int f(t) \, dt + C \\
    u(x) &= 0, \quad x = G \\
    u''(x) &= 0
\end{align}

\begin{equation}
    2(f) = C_1 + C_2
\end{equation}

\begin{align}
    u_{xx} &= 0 \\
    u(x, y) &= 0 \\
    u_x &= 0 \\
    u_x &= F(y) \\
    u(x, y) &= X F(y) + G(x)
\end{align}

% --- Page 8 ---




\section*{Example 2}

\begin{align}
    U_x X + U &= 0, \\
    u'' + u &= 0.
\end{align}

The solution for \( u(x, y) \) is given by:
\[
u(x, y) = C_1(y) \cos x + C_2(y) \sin x
\]

\section*{Example 3}

\begin{align}
    U_{xy} &= 0, \\
    u_{xx} &= 0, \\
    u_x &= F(x).
\end{align}

The general solution is:
\[
u = \int F(x) \, dx + H(x) + G(y)
\]

% --- Page 9 ---


\clearpage

\section*{List of Calculus Facts}

\begin{enumerate}
    \item Partial derivatives are local.
    
    \item $U_{xy} = K_{yx}$
    
    \item Chain rules:
    \begin{align*}
        f(g(x)) &= f'(g(x)) \cdot g'(x) \\
        &= f'(g(x)) \cdot g'(x)
    \end{align*}
    This is a special case of the chain rule. Assume $f = f(y_1, y_2, \ldots, y_n)$ where $y_i = g_i(x_1, \ldots, x_k)$ in the most general form of the chain rule.
    
    \item Green's formula (later):
    \[
    \int_{a}^{b} f(x) \, dx = F(b) - F(a)
    \]
    
    \item 
    \begin{align*}
        I(t) &= \int_{a(t)}^{b(t)} f(x, t) \, dx \\
        I'(t) &= f(b(t), t) \cdot b'(t) - f(a(t), t) \cdot a'(t) + \int_{a(t)}^{b(t)} \frac{\partial f}{\partial t}(x, t) \, dx
    \end{align*}
\end{enumerate}

% --- Page 10 ---




\section*{Jacobian}

Consider the function \( F : \mathbb{R}^n \to \mathbb{R}^m \), where

\[
F = (F_1, F_2, \ldots, F_m)
\]

\section*{Infinite Series}

\emph{(to be covered later)}

\section*{Directional Derivative (Geometry)}

The directional derivative of a function \( u(x, y) \) in the direction of a vector \(\mathbf{u} = (u_1, u_2)\) is given by:

\[
\frac{\partial u}{\partial x} u_1 + \frac{\partial u}{\partial y} u_2
\]

This represents the rate of change of the function \( u \) in the direction of the vector \(\mathbf{u}\).

\section*{Familiar with Math 3D}

\emph{(to be covered later)}

% --- Page 11 ---




\section*{Example 1: Transport Equations}

Consider the transport equation:

\begin{equation}
    a u_x + b u_y = 0
\end{equation}

where \(a\) and \(b\) are constants.

\subsection*{Method 1: Geometry / Gradient of \(u\)}

The gradient of \(u\) is given by:

\[
(a, b) \cdot (u_x, u_y) = -b \cdot a
\]

The equation for the line is:

\begin{equation}
    bx - ay = c
\end{equation}

The solution is:

\[
u(x, y) = F(bx - ay)
\]

To check:

\[
u_x = F' \cdot b, \quad u_y = F' \cdot (-a)
\]

Substituting back into the transport equation:

\[
a u_x + b u_y = a \cdot F' \cdot b + b \cdot F' \cdot (-a) = 0
\]

\subsection*{Method 2: Change of Variables}

Let:

\[
x' = ax + by, \quad y' = bx - ay
\]

% If a sketch is needed, it can be added here using TikZ.

% --- Page 12 ---




\section*{Example 2}

\begin{align}
    u_x &= Hy - a + Hyb = a l x + b u_y, \\
    My &= u_y b + Hy(-a) = b u_y - a n y, \\
    0 &= a l x + b u_y = (a + b) u_x = Hx = 0, \\
    x &= F(y) = F(bx - ay).
\end{align}

\begin{align}
    u_x + y u_y &= 0, \\
    2 &= + y^2 u = 0.
\end{align}

\begin{align}
    (u_x, u_y) &= 0, \\
    (1, \%) &- y = + y = c e^y.
\end{align}

% --- Page 13 ---




\section*{Equations Along the Curve}

Consider the following equations along the curve:

\begin{equation}
u(x, Ex + x) = 0
\end{equation}

\begin{equation}
u(x, x + c) = U_{10}
\end{equation}

\begin{equation}
F(x) = F(y - Ex) = F(bx - ay)
\end{equation}

% --- Page 14 ---



\clearpage
\section*{Review}

\subsection*{Example}

\begin{align}
    ax + by &= 0, \\
    a + b &= 70.
\end{align}

If we consider the equation:
\begin{align}
    y &= Hx + My, \\
    y &= ux + My = 0.
\end{align}

We have:
\begin{align}
    u(x, x + c) &= u(0, c), \\
    u(x, y) &= u(0, c) = u(0, y - tx) = F(y - Ex) = G(ay - bx) = H(bx - ay).
\end{align}

% --- Page 15 ---




\section*{Example 2}

Consider the differential equation:

\begin{equation}
    u_x + y u_y = 0.
\end{equation}

We have the following conditions:

\begin{align}
    y' &= -y, \\
    y &= ce^{-x}, \\
    c &= y e^x, \\
    u(x, ce^{-x}) &= u(0, c) = u(0, y e^x) = F(y e^x).
\end{align}

\section*{Example ?}

In the previous example, if in addition \( u(0, y) = 3 \), then what is the solution?

\section*{Solution}

The solution is given by:

\begin{equation}
    u(x, y) = F(y e^x) = u(0, y) = F(y) = 3.
\end{equation}

% --- Page 16 ---




\section*{Solution}

The solution is given by
\[
u(x, y) = F(ye) = (ye - x)^3
\]

\subsection*{Integrating Factor}

Consider the integrating factor:
\[
e^{y} = e^{-y} (y' - y) = 0
\]

Thus, we have:
\[
e^{y} + y = y
\]

\[
x f_y(y) = x + C
\]

\[
f(y) = x + C
\]

\[
\frac{1}{y} = e^{y}
\]

\[
e^{y} = e^{x} + 6
\]

The missing solution is:
\[
Y = 0
\]

\[
y = c e^{x}
\]

% --- Page 17 ---




\section*{Example 4}

Consider the partial differential equation:
\[
U_x + 2x U_y = 0.
\]

\subsection*{Solution}

We start by considering the characteristic equations:
\[
\frac{dy}{dx} = \frac{2x}{1} \quad \Rightarrow \quad y = x^2 + C.
\]

Thus, the general solution can be expressed as:
\[
u(x, y) = f(y - x^2),
\]
where \( f \) is an arbitrary function.

Given the initial condition \( u(0, -5) = 0 \), we substitute into the general solution:
\[
u(0, -5) = f(-5 - 0^2) = f(-5) = 0.
\]

Therefore, the specific solution satisfying the initial condition is:
\[
u(x, y) = f(y - x^2),
\]
where \( f(-5) = 0 \).

\subsection*{Conclusion}

The solution to the partial differential equation is determined by the function \( f \) which satisfies the initial condition. The characteristic curves are given by \( y = x^2 + C \), and the solution is constant along these curves.

% --- Page 18 ---


\clearpage

\section*{Implicit Differentiation}

Consider the equation:

\begin{equation}
1 + xy = 0.
\end{equation}

The function \( u(x, y) \) is not defined on the curve \( y = -y \).

In general, if \( a(x, y)(x + b(x, y))y = 0 \), the ordinary differential equation (ODE) is:

\begin{equation}
y = \ldots
\end{equation}

Assume \( y = Y(x) \) is implicitly defined by the equation \( U(x, y) = 0 \). Then:

\begin{equation}
u(x, y(x)) = 0.
\end{equation}

Differentiating implicitly, we have:

\begin{align}
u_x + y' u_y &= 0, \\
y' &= -\frac{u_x}{u_y}.
\end{align}

This is equivalent to:

\begin{equation}
S \cdot Nx + y = 0.
\end{equation}

\begin{equation}
y = \ldots
\end{equation}

% --- Page 19 ---




\section*{Exercise 7 of Section 1.2}

Solve the differential equation:
\begin{equation}
    Y U_x + X y = 0
\end{equation}
with the initial condition:
\begin{equation}
    x(0, y) = g
\end{equation}

Determine in which region of the initial problem (IP) the solution is uniquely determined.

\subsection*{Solution}

Consider the differential equation:
\begin{equation}
    y' = y
\end{equation}
which implies:
\begin{equation}
    y = x + c
\end{equation}
where \( c \) is a constant. Rewriting, we have:
\begin{equation}
    y - x = c
\end{equation}

If \( c = 0 \), then:
\begin{equation}
    y = x
\end{equation}

If \( c \neq 0 \), then:
\begin{equation}
    y^2 - x = 1
\end{equation}

If \( c = 0 \), then:
\begin{equation}
    y = x
\end{equation}

If \( c \neq 0 \), then:
\begin{equation}
    y = \ln U(x)
\end{equation}
where \( U(x) = u(x, t) = e \).

\begin{tikzpicture}
    \draw[->] (-1,0) -- (3,0) node[right] {$x$};
    \draw[->] (0,-1) -- (0,3) node[above] {$y$};
    \draw[domain=-1:2, smooth, variable=\x, blue] plot ({\x}, {\x}) node[right] {$y=x$};
    \draw[domain=-1:2, smooth, variable=\x, red] plot ({\x}, {sqrt(\x+1)}) node[right] {$y^2-x=1$};
\end{tikzpicture}

% --- Page 20 ---


\clearpage

\section*{Mathematical Physical Equations}

\subsection*{Wave Equations}

\begin{align*}
    u(x, y) &= e^{\ldots} \\
    y &= x \ldots
\end{align*}

On the other hand, in the green region, we have
\begin{align*}
    y - x &= c < 0 \\
    x &= 1
\end{align*}

\begin{align*}
    u(1, y) &= u(\ldots, 0) = u(\ldots, 0)
\end{align*}

In the blue region,
\begin{align*}
    u(x, y) &= F(y)
\end{align*}

In the green region, sometimes
\begin{align*}
    F &= G
\end{align*}

\subsection*{Heat Equations (Parabolic Equations)}

\subsection*{Laplace Equations (Elliptic Equations)}

\begin{align*}
    a, b, c, \ldots
\end{align*}

% --- Page 21 ---


\clearpage

\section*{Chapter 2: Wave and Diffusion}

\subsection*{2.1 The Wave Equation}

Assume \( U = U(x, t) \) and \( U_{tt} = c^2 U_{xx} \) with \( c > 0 \).

\textbf{Theorem:} The general solutions of the wave equation are given by
\[
u(x, t) = f(x + ct) + g(x - ct),
\]
where \( f \) and \( g \) are arbitrary functions.

\textbf{Proof:}

Assume
\[
U_{tt} - c^2 U_{xx} = 0.
\]

Let \( V = U_t + c U_x \) and \( W = U_t - c U_x \). Then,
\[
V_t - c V_x = 0 \quad \text{and} \quad W_t + c W_x = 0.
\]

The solutions to these equations are
\[
v(x, t) = h(x + ct) \quad \text{and} \quad w(x, t) = k(x - ct),
\]
where \( h \) and \( k \) are arbitrary functions.

Thus, the general solution is
\[
u(x, t) = f(x + ct) + g(x - ct).
\]

% --- Page 22 ---




\section*{Wave Equation Solutions}

\subsection*{General Solution of the Homogeneous Equation}

Find the general solution of the homogeneous equation, which is given by:

\begin{equation}
    u_{tt} + c^2 u_{xx} = 0
\end{equation}

The solution can be expressed as:

\begin{equation}
    u = g(x - ct)
\end{equation}

\subsection*{Particular Solution}

Find a particular solution by assuming:

\begin{equation}
    u = f(x + ct)
\end{equation}

Then, we have:

\begin{align}
    u_t &= c f'(x + ct) \\
    u_x &= f'(x + ct)
\end{align}

Substituting into the wave equation:

\begin{equation}
    u_{tt} + c^2 u_{xx} = c^2 f''(x + ct) = h(x + ct)
\end{equation}

Thus, we find:

\begin{equation}
    f(x) = \int h(x) \, dx
\end{equation}

The general solution of the wave equation is:

\begin{equation}
    u = g(x - ct) + f(x + ct)
\end{equation}

\subsection*{Alternative Method: Change of Variables}

Alternatively, we can solve the equation by changing variables. Let:

\begin{align}
    \xi &= x + ct \\
    \eta &= x - ct
\end{align}

Then, the derivatives transform as follows:

\begin{align}
    u_t &= c u_\xi - c u_\eta \\
    u_x &= u_\xi + u_\eta
\end{align}

This transformation simplifies the wave equation to:

\begin{equation}
    u_{\xi \eta} = 0
\end{equation}

The solution in terms of the new variables is:

\begin{equation}
    u(\xi, \eta) = F(\xi) + G(\eta)
\end{equation}

where \( F \) and \( G \) are arbitrary functions determined by initial conditions.

% --- Page 23 ---


\clearpage

\section*{Equations and Solutions}

\begin{align}
    U_g &= U_{se} + U_s = c(U_{ss} - U_{su}) \\
    U_n &= U_g + U_{ng} = c(U_{gy} - U_{nn}) \\
    U_H &= 2(U_{es} + U_{ng}) - 2cU_{gy} \\
    U_x &= 4 + Y_y \\
    U_{xx} &= U_{gg} + U_{ny} + 213
\end{align}

\begin{equation}
    0 = U_H - in_x = -45U_{gy}
\end{equation}

\begin{equation}
    \text{KeyO: } \Theta U_g() = \text{fixtc} + \text{gla}
\end{equation}

\begin{equation}
    U_H - cU_{xx} = 0
\end{equation}

\begin{equation}
    u(x, 0) = f(x)
\end{equation}

\begin{equation}
    u_f(x, 0) = 4(X)
\end{equation}

\section*{Solution}

\begin{equation}
    u(x, t) = f(x + t) + g(x - ct)
\end{equation}

Let \( t = 0 \), then

\begin{equation}
    u(x, 0) = f(x) + f(x)
\end{equation}

\begin{equation}
    U^+(x, t) = cf'(x + t) - cg(x - t)
\end{equation}

% --- Page 24 ---


\clearpage

\section*{Equations and Transformations}

\begin{align}
    4(x) &= (x + 0) = c f(x) - 38'(x) \\
    \int (y(x)) \, dx &= f(x) - g(x) \\
    f + y &= 4 \\
    f - g &= \int 4(x) \, dx \\
    f &= k + \int t(y(x)) \, dx \\
    g &= d - \int t(4(x)) \, dx
\end{align}

\subsection*{Rewrite}

\begin{align}
    f(x) &= za(x) + So \\
    g(x) &= 1P(x) - E
\end{align}

\subsection*{Function Transformation}

\begin{align}
    u(x +) &= f(x + H) + g(x - t) \\
    &= \left( d(x + t) + q(x - Ct) \right) + y(s) \, ds
\end{align}

\subsection*{d'Alembert Formula}

\begin{equation}
    \int \left( d(x + t) + q(x - Ct) \right) + y(s) \, ds
\end{equation}

% --- Page 25 ---


\clearpage

\section*{Wave Equation and d'Alembert's Solution}

Assume that \( u = u(x, t) \) is a function of \( x \) and \( t \). The wave equation is given by

\begin{equation}
    u_{tt} - c^2 u_{xx} = 0.
\end{equation}

\subsection*{General Solution}

The general solution is given by

\begin{equation}
    u(x, t) = f(x + ct) + g(x - ct),
\end{equation}

where \( f \) and \( g \) are functions of one variable.

\subsection*{Initial Value Problem}

For the initial value problem

\begin{align}
    u_{tt} - c^2 u_{xx} &= 0, \\
    u(x, 0) &= \phi(x), \\
    u_t(x, 0) &= \psi(x),
\end{align}

we have the d'Alembert formula

\begin{equation}
    u(x, t) = \frac{1}{2} \left[ \phi(x + ct) + \phi(x - ct) \right] + \frac{1}{2c} \int_{x-ct}^{x+ct} \psi(s) \, ds.
\end{equation}

% --- Page 26 ---




\section*{Example 1}

Assume $\phi(x) = 0$ and $\psi(x) = \cos x$. Then

\begin{align*}
    u(x, t) &= \frac{1}{2} \int_{x-t}^{x+t} \psi(s) \, ds \\
    &= \frac{1}{2} \int_{x-t}^{x+t} \cos s \, ds \\
    &= \frac{1}{2} \left[ \sin(x+t) - \sin(x-t) \right] \\
    &= \sin x \cos t + \cos x \sin t.
\end{align*}

\section*{Example}

The Plucked String

Assume $\phi(x) = 0$ and $\psi(x) = 1$. Then

\begin{align*}
    u(x, t) &= \frac{1}{2} \left[ \phi(x+ct) + \phi(x-ct) \right] + \frac{1}{2c} \int_{x-ct}^{x+ct} \psi(s) \, ds \\
    &= \frac{1}{2c} \int_{x-ct}^{x+ct} 1 \, ds \\
    &= \frac{1}{2c} \left[ (x+ct) - (x-ct) \right] \\
    &= t.
\end{align*}

% --- Page 27 ---




\section*{Example 3/Ex 11}

Find the general solution of 
\begin{equation}
    3u'' + 10(x + 3)u = \sin(x + \pi).
\end{equation}

\subsection*{Solution}

The linear operator is 
\begin{equation}
    \mathcal{L} = 3\frac{d^2}{dx^2} + 10(x + 3).
\end{equation}

So the equation can be written as 
\begin{equation}
    \mathcal{L}u = \sin(x + \pi).
\end{equation}

To find a particular solution, we assume 
\begin{equation}
    u_p(x) = A \sin(x + \pi) + B \cos(x + \pi).
\end{equation}

% --- Page 28 ---


\clearpage

\section*{General Solutions of Differential Equations}

We need to find general solutions of the differential equation \( Lu = 0 \).

\begin{align}
    UH - U_{xx} &= 0, \\
    3UH + 10U_x + 3U_{xx} &= 0.
\end{align}

Let \( v = (3 + 2)u \). Then we have:

\begin{align}
    30 + U_x &= 0, \\
    v &= f(3x - t), \\
    u &= g(3t - x) + H(3x - t).
\end{align}

% --- Page 29 ---




\section*{Causality and Energy}

\begin{align}
    -T U_{xx} &= 0
\end{align}

where \( f, T \) are constants.

If we take \( c = E \), then

\begin{align}
    -Y U_{x} &= 0
\end{align}

Define Kinetic Energy \( E_{\text{mr}} \) as follows:

\begin{align}
    \text{KE} &= \int U \, dy \\
    &= \int U E \, dx \\
    &= \int (f U) \, dx \\
    &= p(u + u + dx) \\
    &= u + u_{xx} \, dx \\
    &= \int u + du_{x} \\
    &= T T u_{x} \\
    &= -T(U_{x} U_{x} + dX)
\end{align}

% --- Page 30 ---




\section*{Potential Energy and Mechanical Energy Conservation}

The potential energy is denoted by \( U \). Then the total energy, which is the sum of kinetic energy (KE) and potential energy (PE), is given by:

\[
\text{Total Energy} = \text{KE} + \text{PE} = 0
\]

This defines the mechanical energy to be:

\[
\text{Mechanical Energy} = \text{KE} + \text{PE}
\]

Thus, we have the mechanical energy conservation equation:

\[
\frac{\partial U}{\partial t} + \frac{\partial U}{\partial x} = 0
\]

with boundary conditions:

\[
u(0) = q(x) = b
\]

and

\[
U(x, 0) = 0
\]

The energy equation can be expressed as:

\[
E = \int \left( \frac{1}{2} y u p + \frac{\partial U}{\partial x} \right) \, dx
\]

% --- Page 31 ---


\clearpage

\section*{Mathematical Expressions}

\begin{align*}
    T(u) &= \int u(x) \, dx \\
    a + b &= o \\
    I + \int d(x) \, dx &= b \\
    -b &< x < a \\
    4(x) &= 2 \\
    b + t &= -a \\
    x_0 &= 12a \\
    P(x) &= L \\
    -a &< x < 0 \\
    |x| &< a \\
    2b^2 \\
    \int k \, dx &= 2 \\
    u(x) &= d(x) + t \\
    u(x) &= c + x \\
    E(f) &= KE + PE \\
    &= \int (i + u(x)) \, dx
\end{align*}

% --- Page 32 ---




\begin{align}
    U_t &= z(k'(x + c^+) - k'(x - t)), \\
    J_u F &= \int (k'(x + c^+) - q'(x - c^+)) \, dx, \\
    J_u x &= \int (d'(x + c^+) + q'(x + c)) \, dx, \\
    E(t) &= \int (d'(x + c^+) + q'(x - c)) \, dx \\
    &= \int (k'(x + c^+) - c) (x - t) \, dx, \\
    1'(x + c^+) \, dx &= H, \\
    &'(y) \, dy = i, \\
    E(CH) &= 2c \left( \int d \, dy - A \right).
\end{align}

% --- Page 33 ---


\clearpage

\section*{The Diffusion Equation}

\subsection*{Equation}
\begin{equation}
    u_t = R u_{xx}
\end{equation}
where \( u(x, t) \) is a function on the rectangular domain \([0, e] \times [0, T]\).

\subsection*{Maximum Principle}

\subsubsection*{Theorem (Strong Maximum Principle)}
The maximum value of \( u(x, t) \) can only be reached on the lines:
\begin{align*}
    t &= 0, \\
    x &= 0, \\
    x &= e, \\
    t &= T.
\end{align*}

\subsubsection*{Theorem (Weak Maximum Principle)}
\begin{equation}
    \max u(x, t) = \max u(x, 0)
\end{equation}

% --- Page 34 ---




\section*{Proof}

What if at some point in the interior of $\mathbb{R}$ that reaches the maximum value of $U(x, y)$, then $(x_0, y_0)$ satisfies $U_x(x_0, y_0) = 0$.

The Hessian matrix is given by:

\[
\begin{bmatrix}
U_{xx} & U_{xy} \\
U_{xy} & U_{yy}
\end{bmatrix}
\]

For a maximum, the Hessian must be negative definite, which is equivalent to:

\[
U_{xx} < 0 \quad \text{and} \quad U_{xx}U_{yy} - U_{xy}^2 > 0
\]

At $(x_0, y_0)$, we have:

\[
U_{xx}U_{yy} - U_{xy}^2 > 0
\]

Let $v(x, t) = u(x, t) + 2x^2$. Then:

\[
u_x = R U_x = R U_{xx} - 2kE
\]

And:

\[
U_{xx} + 2k
\]

This leads to a contradiction if $U(x_0, y_0) \neq 0$.

% --- Page 35 ---




\section*{Assumptions and Maximum Conditions}

Assume $(X_0, Y_0)$ is an interior point such that $I$ reaches a maximum. Then

\begin{align}
    U_t(X_0, Y_0) &= 0, \\
    U_{xx}(X_0, Y_0) &\neq 0.
\end{align}

This leads to a contradiction:

\begin{equation}
    0 = V(X_0, Y_0) = R V_{xx} - 2k^2 - 2\delta s_0.
\end{equation}

\section*{Maximum Value Analysis}

Let $(X_0, Y_0)$ be the maximum point of $v$. Then

\begin{equation}
    \text{Max } v(x, t) = \text{Max } u(x, t) + 2.
\end{equation}

Thus, we have

\begin{align}
    U(x, t) &= v(x, t) - E x, \\
    \text{Max } u(x, t) &< \text{Max } U(x, t).
\end{align}

Finally, we conclude

\begin{equation}
    R = \text{Max } v(x, t) - \text{Max } U_t + 3.
\end{equation}

% --- Page 36 ---




\section*{Uniqueness of the Diffusion Equation}

\subsection*{Theorem}

Consider the diffusion equation:

\begin{align*}
    u_t &= R u_{xx}, \\
    u(x,0) &= f(x), \\
    u(0,t) &= g(t), \\
    u(l,t) &= h(t).
\end{align*}

Then the solution is unique.

\subsection*{Proof}

Let \( u_1 \) and \( u_2 \) be two solutions. Define

\[
v(x,t) = u_1(x,t) - u_2(x,t).
\]

Then \( v(x,t) \) satisfies:

\begin{align*}
    v_t &= R v_{xx}, \\
    v(x,0) &= 0, \\
    v(0,t) &= 0, \\
    v(l,t) &= 0.
\end{align*}

The maximum principle implies that

\[
\max v(x,t) = \max v(x,0) = 0.
\]

Thus, \( v(x,t) = 0 \) for all \( x \) and \( t \), which implies \( u_1(x,t) = u_2(x,t) \). Therefore, the solution is unique.

% --- Page 37 ---




\section*{Uniqueness Theorem for Wave Equation}

Consider the wave equation:

\begin{equation}
    U_{tt} - c^2 U_{xx} = 0
\end{equation}

with initial conditions:

\begin{align}
    u(x, 0) &= 0, \\
    u_t(x, 0) &= 0.
\end{align}

We define the energy function \( E(t) \) as:

\begin{equation}
    E(t) = \int \left( U_t^2 + c^2 U_x^2 \right) \, dx
\end{equation}

At \( t = 0 \), the energy is:

\begin{equation}
    E(0) = 0
\end{equation}

Since the energy \( E(t) \) is constant over time, we have:

\begin{equation}
    E(t) = \text{const} = 0
\end{equation}

Thus, \( U_t = 0 \) and \( U_x = 0 \), implying that the solution is unique and zero everywhere.

\end{document}
