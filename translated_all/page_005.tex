```latex
\section*{Linear and Nonlinear Equations}

In linear algebra, a transformation \( T: V \to W \) is called a linear transformation if:

\begin{align}
    T(u + v) &= T(u) + T(v), \\
    T(au) &= aT(u)
\end{align}

or equivalently,

\begin{equation}
    T(au + bv) = aT(u) + bT(v)
\end{equation}

\subsection*{Definition}

An operator is called linear if:

\begin{align}
    L(u + v) &= L(u) + L(v), \\
    L(au) &= aL(u)
\end{align}

A differential operator is linear if:

\begin{align}
    L(u + v) &= L(u) + L(v), \\
    L(au) &= aL(u)
\end{align}

\subsection*{Example}

Consider the differential equation:

\begin{equation}
    u_{xx} + u_{yy} = 0
\end{equation}

This is an example of a linear partial differential equation (PDE).

The transport equation is given by:

\begin{equation}
    u_x + u_y = 0
\end{equation}

A linear PDE can be written in the form:

\begin{equation}
    Lu = f(x)
\end{equation}

where \( L \) is a linear operator, and \( f(x) \) is a given function.

The system of linear equations can be represented as:

\begin{equation}
    Ax = b
\end{equation}
```
